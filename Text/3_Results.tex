% Image sizes:
\newcommand\subSizeO{0.43}
\newcommand\subSizeI{0.9}
\newcommand\figSize{0.7}

\subsection{Flow rates}
The green solution flowed faster through the paper than the red. Both solutions flowed faster along the fibers than against them. A graph showing absorption distance versus time can be seen in figure \ref{fig:diff}.

% Diffusion plot
\begin{figure}[H]
    \centering
    \includegraphics[width=.8\linewidth]{Images/plot_diff.png}
    \caption{Graph of absorption distance versus time for the green and red solutions..}
    \label{fig:diff}
\end{figure}

\subsection{Mixing}
As can be seen in figure \ref{fig:y} some mixing occurred along the laminated flows. The added piece across the circuit seen in figure \ref{fig:sub:y-} also seem to increase the mixing. A good flow between the pieces was observed, but the two solutions continued the divide and flowed at different directions.

% Y
\begin{figure}[H]
\centering
\begin{subfigure}[b]{\subSizeO\textwidth}
  \centering
  \includegraphics[width=\subSizeI \linewidth]{Images/Straight.jpg}
  \caption{Y-shaped circuit.}
  \label{fig:sub:y}
\end{subfigure}%
\begin{subfigure}[b]{\subSizeO \textwidth}
  \centering
  \includegraphics[width=\subSizeI \linewidth]{Images/Straight_peice.jpg}
  \caption{Y-shaped circuit with a piece across it.}
  \label{fig:sub:y-}
\end{subfigure}
\caption{The resulting flow of solutions of the two tests with Y-shaped circuits.}
\label{fig:y}
\end{figure}

As can be seen in figure \ref{fig:jagged} the circuit with jagged edges did increase the mixing, but the flow of the green solution was seriously halted.

% Jagged
\begin{figure}[H]
    \centering
    \includegraphics[width=\figSize \linewidth]{Images/Jagged2.jpg}
    \caption{The resulting flow in the circuit with jagged edges.}
    \label{fig:jagged}
\end{figure}

From figure \ref{fig:hourglass} can be seen that the hour glass shaped design did not increase mixing.

%Hour glass
\begin{figure}[H]
    \centering
    \includegraphics[width=\figSize \linewidth]{Images/hour_glass.jpg}
    \caption{The resulting flow in the circuit with a central channel shaped like a hour glass.}
    \label{fig:hourglass}
\end{figure}

The 3-layered mixer did increase mixing significantly at firsts, as can be seen in figure \ref{fig:sub:3mixt1}. However after some further time the color seemed two start to separate again.

% 3 layer mixer
\begin{figure}[H]
\centering
\begin{subfigure}[b]{\subSizeO\textwidth}
  \centering
  \includegraphics[width=\subSizeI \linewidth]{Images/t1_3laymix.png}
  \caption{After \SI{1}{\minute}.}
  \label{fig:sub:3mixt1}
\end{subfigure}%
\begin{subfigure}[b]{\subSizeO \textwidth}
  \centering
  \includegraphics[width=\subSizeI \linewidth]{Images/t1_3laymix.jpg}
  \caption{After \SI{2}{\minute}.}
  \label{fig:sub:3mixt2}
\end{subfigure}
\caption{The resulting flows at two times in the 3-layered mixing circuit.}
\label{fig:y}
\end{figure} 

After dripping green and red solution on a paper the solutions initially mixed. They did however divide in their flow outwards forming rings of differing colors. After the liquids had dispersed though out the whole paper the dividing continued, patches of green and red color had formed.

\subsection{Directing flow with differing number of inputs and outputs}
When instead studying directing flow, number of outputs, inputs and size was important. As seen below in \ref{fig:diamon} different dyes have different flow rates, which resulted in the green dye taking over before reaching the diamond shape. However, after the diamond the dyes take up, almost, half of the space each.

\begin{figure}[H]
    \centering
    \includegraphics[width=0.4 \linewidth]{Images/direct.flow/diamond2.jpg}
    \caption{The resulting flow in the circuit with a diamond-shaped section.}
    \label{fig:diamon}
\end{figure}
 
 Next we looked at two outlets with different shaped middle sections. In \ref{fig:sub:straight} is the straight paper-channel, whereas in \ref{fig:sub:loop} the looped version can be seen. The loop made it harder for the two dyes to stay separated during the flow in the loop-section. However, when reaching the outlets the fluids divided well and went separate ways, to different outlets. In \ref{fig:sub:straight} we can instead see a very clear line down the middle of the whole paper. However, the red got absorbed down at the very bottom of the green outlet.
\begin{figure}[H]
\centering
\begin{subfigure}[b]{\subSizeO\textwidth}
  \centering
  \includegraphics[width=\subSizeI \linewidth]{Images/direct.flow/straight90deg.jpg}
  \caption{Straight cutout with perpendicular outlets.}
  \label{fig:sub:straight}
\end{subfigure}%
\begin{subfigure}[b]{\subSizeO \textwidth}
  \centering
  \includegraphics[width=\subSizeI \linewidth]{Images/direct.flow/loopchannel.jpg}
  \caption{Looped cutout with perpendicular outlets.}
  \label{fig:sub:loop}
\end{subfigure}
\caption{The resulting flows for a straight and a looped circuit, using dye.}
\label{fig:straightloop}
\end{figure} 

Instead of only using two outlets, looking at \ref{fig:3out} we can see that the dye divided itself into the three outlets, with a clear difference between the fluids.
\begin{figure}[H]
    \centering
    \includegraphics[width=0.5 \linewidth]{Images/direct.flow/3outlet.jpg}
    \caption{The resulting flow in the circuit with two inlets and three outlets.}
    \label{fig:3out}
\end{figure}


Below, in \ref{fig:3in} we see the result of testing three inlets and outlets. Here we varied the color of the outer two inlets relatively the middle inlet. When the blue dye is in the middle inlet, see \ref{fig:sub:3in1}, it has nice, straight lines with big contrast between the dyes. However, when the red dye (the slower one) is in the middle, see \ref{fig:sub:3in1}, the lines are not as straight. 
\begin{figure}[H]
\centering
\begin{subfigure}[b]{\subSizeO\textwidth}
  \centering
  \includegraphics[width=\subSizeI \linewidth]{Images/direct.flow/3in-3out.jpg}
  \caption{Red dye on the outer inlets.}
  \label{fig:sub:3in1}
\end{subfigure}%
\begin{subfigure}[b]{\subSizeO \textwidth}
  \centering
  \includegraphics[width=\subSizeI \linewidth]{Images/direct.flow/3in2.jpg}
  \caption{Blue dye on the outer inlets.}
  \label{fig:sub:3in2}
\end{subfigure}
\caption{The resulting flows of using 3 inlets and 3 outlets, with the outer 2 inlets being one dye and the middle inlet another.}
\label{fig:3in}
\end{figure}

After testing three inlets, we built a more complex circuit as seen in \ref{fig:3D}. The inlet with the blue dye splits up and goes over the middle section where the two dyes meet. The red dye was fast enough to travel up and meet the blue dye up in the loop, which stopped both fluids from spreading further in the loop. 
\begin{figure}[H]
    \centering
    \includegraphics[width=0.5 \linewidth]{Images/direct.flow/3D.jpg}
    \caption{The resulting flow in the circuit with an elevated loop above the rest of the circuit.}
    \label{fig:3D}
\end{figure}

Lastly, we looked at what happened when we had many outlet and branches. In \ref{fig:sub:branches} the dyes clearly followed the papercutouts, however the blue dye was faster and climbed up one or two branches of the red dye. To make sure the dyes could not use an outlet for another dye we used the laminated, layered circuit, seen in \ref{fig:sub:u}. The capillary forces made the dyes climb up all the way to the top where we allowed the papers to meet. Only when the papers, of the different dyes, where pushed together the dyes would travel across. This resulted in no unwanted spreading of the fluids until it was intended to happen.
\begin{figure}[H]
\centering
\begin{subfigure}[b]{\subSizeO\textwidth}
  \centering
  \includegraphics[width=\subSizeI \linewidth]{Images/direct.flow/manybranches.jpg}
  \caption{Many outlets, some overlapping.}
  \label{fig:sub:branches}
\end{subfigure}%
\begin{subfigure}[b]{\subSizeO \textwidth}
  \centering
  \includegraphics[width=\subSizeI \linewidth]{Images/direct.flow/up.jpg}
  \caption{Layered circuit, standing up.}
  \label{fig:sub:u}
\end{subfigure}
\caption{The resulting flows using many branches and outlets.}
\label{fig:manyup}
\end{figure}