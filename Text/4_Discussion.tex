\subsection{Mixing}
The red and green color molecules does not seem to dissolve when in solution in paper despite doing so in just water. It would have been a good idea to test this first. The fact that some mixing did occur with jagged-edge- and 3-layered-designs, indicates that they might be effective for solutions which do mix in paper. Hence it would be interesting to follow up with more tests of these designs with other liquids.

\subsection{Directing flow with differing number of inputs and outputs}
The dyes used were red, blue and green, which all have different sized molecules in the fluid. However, since green is a mixture of blue and yellow, green and blue behaved similarly. They had similar speed relative to the red dye, which was clearly slower than the rest of the dyes. The relative speed of the dye made it difficult to make the fluids flow equally. Important to note though, is that the fact that different fluids have different flow rates in paper, makes it easy to use for diagnostics. In our case, we mostly tried to direct flow in paper, and so different speed was a disadvantage.

In our attempts to direct the flow we tried to divide the fluids and make them go to separate outlets. This did not always happen. For example, looking at \ref{fig:3out} the middle outlet does not contain solely one dye but both dyes. Additionally, the line is not as straight as in \ref{fig:sub:straight}, however, is clearly divided. The dye has not yet reached the end of all outlets and is most likely the cause to the not so straight line and high contrast between the two fluids.
