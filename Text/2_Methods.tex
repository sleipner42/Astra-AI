Three different solutions of distilled water with $10\%$ food coloring was used for the experiments. The three colors used were red, blue and green. These were mixed, using a 5ml syringe and disposable pipettes.

The inlets were dipped in dye at similar times and the test of the circuits were ended as soon as the dye reached the outlets, to eliminate effects caused by diffusion.

\subsection{Flow rates}
To measure the absorption rates of the green and red solutions, clean-room papers, one for each color, were marked with lines every centimeter along two neighbouring sides. This was done to be able to decide the flow rate both along and against the fibers. For each paper the edge between the two marked sides was dipped in one of the color solutions and the time to reach each marked line, measured. 

\subsection{Mixing}
To make functional circuits mixing of different liquids, could be a crucial component. Firstly one layer circuits were tried. Here a scissor was used to cut out a number of different shapes from filter paper. The inlets of these were dipped in the different color solutions. The resulting flow of the liquids was filmed.

Firstly two inputs to one straight section was tested with the green and red solutions, to see how the color molecules diffused. This created a y-shaped circuit. This was redone with a piece of paper laying across the straight section to see if this changed the results and how well the liquids would diffuse between sheets.

Then a shape with two inputs to a straight section with of-set jagged edges on both sides, was tested again with the red and green solutions. 

Next a shapes with two inputs to a time glass shaped straight was tested with the same liquids. This was tried with the intent to force the liquids together to mix and then spread out again to drive the flow.

Lastly a more advanced three layered construction was tried. Here two rectangular pieces was used as inlets, connecting on each side at the same spot of a third rectangular piece creating a central channel.

Finally one drop of the red and one of the green solution were dripped, using a pipette, in the middle of a filer paper at the same time. This was done to study how well the liquids could be mixed in paper.

\subsection{Directing flow with differing number of inputs and outputs}
After attempting mixing, we instead tried to determine and to direct flow is important for analytical tools in paper microfluidics.  Firstly, simple one layer cutouts with few branches and outlets were tested. These inlets and outlets varied in size and the cutouts had different shapes and forms.

The first one that was made had two inlets and one single outlet. However, the section in between was not completely straight. Right below were the inlets met a diamond shape was cut out, forcing the dye to go around and meet again after the diamond shape.

Now two inlets were studied. The inlets were arranged in a in a y-shape, and the cutout had a long section in between the outlets and the inlets with a, approximately, 170$\degree$ loop and two outlets that were perpendicular to the channel.

After studying the loop, the same idea was used, with y-shaped inlets, perpendicular outlets and this time the section between the inlet and outlet was straight. 

Next a shape with, again, two outlets and two inlets were cut out. This time the inlets and outlets all met at a 45 $\degree$ angle from each other. The inlets and outlets were kept approximately the same size and length. 

After studying two outlets, three outlets were now studied. Again, the y-shaped inlet was used, but three outlets that branched out after the two inlets meet, were used instead of two. 

Next shape had tree inlets as well as three outlets, and allowed for looking at hydrodynamic focusing in middle section of the the clean-room paper.

After this, instead of looking at simple one layered structures, a circuit with an elevated loop for one of the dyes was studied. This circuit had two inlets and one outlet, however one of the dyes was partly looped over the other dye and brought back on to the midsection before reaching the outlet.

After looking at few branches, we started making plenty. Equal amount of outlets were cut out for both dyes and the shapes had only one inlet per dye. The branches overlapped on some of the branches and tape was used to separate the paper and there for the flow of dye. 

Lastly, the same idea with many branches was used, however, here the paper was laminated in a laminatormachine only leaving holes for the inlets and outlets. The channels for the dyes were then taped together letting the open outlets touch. The layered circuit was then taped onto the little containers, that were to be filled with the dye, so that the circuit would stand up, allowing us to look at the strong capillary action.


